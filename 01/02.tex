\subsection{%
  Лекция \texttt{23.09.08}.%
}

Уровни архитектуры данных:
\begin{enumerate}
\item
  Внешний.

  Решаем проблему представления данных пользователю (как агрегировать и
  детализировать данные?).

\item
  Концептуальный.

  Занимаемся выделением сущностей, решаем проблемы безопасности (разрешения +
  ограничения). Определяем семантику (например, договариваемся, что кредитный
  рейтинг будет храниться в виде числа от 1 до 5 и сопоставляем каждому
  числу некоторое словесное описание, которое записано в документации, но не в
  самой базе данных).

\item
  Внутренний.

  Решаем проблемы хранения данных на физическом уровне.
\end{enumerate}

Для простоты выделяют три модели данных, которые связаны с некоторыми описанными
выше уровнями архитектуры данных.

\begin{enumerate}
\item
  Модель Сущность--Связь (внешний + концептуальный уровни).

\item
  Логическая (даталогическая модель) модель (все уровни).

\item
  Физическая модель (концептуальный + внутренний уровни).
\end{enumerate}

\subheader{Модель Сущность--Связь}

\begin{definition}
  \textit{Сущность} --- это множество экземпляров (реальных или абстрактных) 
  однотипных объектов предметной области.
\end{definition}

\begin{definition}
  Сущность называется \textit{сильной}, если её экземпляры могут существовать независимо.
  \textit{Слабые} сущности могут существовать только при наличии одного или нескольких
  экземпляров сильной сущности.
\end{definition}

\begin{example}
  Пусть у нас есть две сущности: Студент и Группа. Студент будет сильной
  сущностью, т.к. он может существовать и без группы, а Группа будет слабой
  сущностью, т.к. она не может существовать без студентов.

  Этот пример очень условный: деление сущностей на слабые и сильные зависит от
  обстоятельств. В некоторых случаях Группа вполне может быть сильной сущностью.
\end{example}

Каждая сущность обладает одним или несколькими атрибутами. Атрибуты делятся на
\begin{enumerate}
\item
  Простые (например дата рождения)

\item
  Составные (например адрес, если хранить город, улицу, дом и т.п. отдельно)
\end{enumerate}

\begin{remark}
  Один и тот же атрибут (например, ФИО) может в разных случаях быть как простым
  (если его хранить как одну строчку), так и составным (если отдельно хранить
  фамилию, отдельно имя и отдельно отчество).
\end{remark}

Также атрибуты можно делить по другому принципу:
\begin{enumerate}
\item
  Обязательные (например email в некоторых случаях может быть обязательным
  атрибутом).

\item
  Необязательные (например отчество может быть необязательным атрибутом).
\end{enumerate}

Помимо этого, атрибуты делятся на
\begin{enumerate}
\item
  Однозначные (например дата рождения, она у каждого ровно одна).

\item
  Многозначные (например номер телефона, у кого-то может быть несколько номеров
  телефона).
\end{enumerate}

Для каждого атрибута мы определяем \textit{домен}, т.е. множество допустимых значений.
Доменом может быть перечисление, множество всех натуральных чисел, регулярное
выражение и т.д.

Чтобы показать отношения между сущностями, используются связи. На уровне
Сущность--Связь они обычно несут семантическую нагрузку, например Студент
\textbf{принадлежит} Группе.
Существует три вида связи
\begin{enumerate}
\item 
  Один-к-одному (\(1\):\(1\)).

  Обычно все связи этого вида это искусственно выделенные атрибуты. Зачем же
  тогда нужна эта связь? Рассмотрим на примере. Допустим, у нас есть две
  сущности: Студент и Паспорт. Логично, что у каждого студента один паспорт и у
  каждого паспорта один студент, т.е. связь вида \(1\):\(1\). Почему же нельзя
  добавить паспорт в атрибуты сущности Студент? Так можно сделать, но связь вида
  \(1\):\(1\) может быть удобна в целях безопасности: в таком случае все
  паспорта можно будет хранить отдельно от студентов и (например) дополнительно
  шифровать (шифровка же всей сущности Студент будет мало того, что излишне
  затратной, так еще и не очень нужной).

\item 
  Один-ко-многим (\(1\):\(N\)).

\item 
  Многие-ко-многим (\(N\):\(N\)).
\end{enumerate}
