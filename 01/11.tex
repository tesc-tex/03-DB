\subsection{%
  Лекция \texttt{23.11.10}.%
}

\begin{definition}
  Распределенная база данных~--- набор логически связанных между собой
  разделяемых данных и их описаний, которые физически распределены по нескольким
  вычислительным узлам.
\end{definition}

Фрагментация бывает

\begin{enumerate}
\item
  Горизонтальной (партиционирование, шардирование).

\item
  Вертикальной.
  
  В этом случае сложнее поддерживать целостность.

\item
  Смешанной.
\end{enumerate}

\begin{definition}
  Реплика это множество различных физических копий одного объекта базы данных,
  для которых обеспечивается синхронизация между собой.
\end{definition}

Есть \(3\) стратегии размещения:

\begin{enumerate}
\item
  Раздельное (фрагментированное) размещение.

  Каждый фрагмент существует в единственном экземпляре и лежит на отдельном
  узле.

\item
  Размещение с полной фрагментацией.

  На каждом узле храним полную копию базы данных.

\item
  Размещение с выборочной фрагментацией.

  Для каждого фрагмента определяем нужное количество реплик.
\end{enumerate}

\begin{definition}
  Распределенная СУБД~--- это комплекс программ, предназначенных для управления
  распределенной базой данных и позволяющий сделать распределенность данных
  прозрачной для конечного пользователя.
\end{definition}

\(4\) уровня прозрачности:

\begin{enumerate}
\item
  Прозрачность фрагментации.

\item
  Прозрачность расположения фрагмента.

\item
  Прозрачность количества реплик.

\item
  Прозрачность контроля доступа.
\end{enumerate}

Распределенные базы данных делятся на гомогенные (используется одна и та же
СУБД) и гетерогенные (используются разные СУБД и, возможно, даже разные модели
данных).

\(12\) правил Дейта для распределенной СУБД:

\begin{enumerate}
\item
  Правило локальной автономности.

  Локальные данные принадлежат локальным владельцам и сопровождаются локально.

\item
  Отсутствие опоры на центральный узел.

  В системе не должно быть ни одного узла, без которого она не сможет
  функционировать.

\item
  Непрерывное функционирование.

\item
  Независимость от расположения.

\item
  Независимость от фрагментации.

\item
  Независимость от репликации.

\item
  Поддержка обработки распределенных запросов.

\item
  Поддержка обработки распределенных транзакций.

\item
  Независимость от типа оборудования.

\item
  Независимость от сетевой архитектуры.

\item
  Независимость от операционной системы.

\item
  Независимость от типа СУБД.
\end{enumerate}

Для поддержки распределенных запросов и транзакций необходимо знать:

\begin{enumerate}
\item
  К какому фрагменту нужно обратиться?

\item
  Какую копию фрагмента использовать (если это репликация)?

\item
  Какое из местоположений нужно использовать для построения временных структур?
  Какие данные как хранить и как их перемещать?
\end{enumerate}
